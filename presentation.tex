\documentclass{beamer}
\usetheme{Warsaw}
\usepackage[utf8]{inputenc}
\usepackage[T1]{fontenc}
\usepackage{mathtools}


\title[DNNLM]{Deep Neural Network Language Models}
\author[A. Peyrard, T. Cantenot]{Alex Peyrard, Thierry Cantenot}
\institute{Shanghai JiaoTong University}
\date{\today}

\setbeamertemplate{blocks}[rounded][shadow=false]
\addtobeamertemplate{block begin}{\pgfsetfillopacity{0.8}}{\pgfsetfillopacity{1}}
\setbeamercolor*{block title example}{fg=blue!150, bg= blue!10}
\setbeamercolor*{block body example}{fg= black, bg= blue!5}

\addtobeamertemplate{footline}{\insertframenumber/\inserttotalframenumber}


\begin{document}
\begin{frame}[plain]
	  \titlepage
\end{frame}

\begin{frame}{Introduction}
	\begin{center}
		Deep Neural Networks Language Models\\
		Ebru Arısoy, Tara N. Sainath, Brian Kingsbury, Bhuvana Ramabhadran\\
		IBM T.J. Watson Research Center\\
		Yorktown Heights, NY, 10598, USA\\
		\{earisoy, tsainath, bedk, bhuvana\}@us.ibm.com
	\end{center}
\end{frame}

\begin{frame}{What is a Language Model?}
	\begin{quote}
		A statistical language model assigns a probability to a sequence of m words $P(w_1,\ldots,w_m)$ by means of a probability distribution. \\ \flushright\emph{Wikipedia -- Language Models}
	\end{quote}
\end{frame}

\begin{frame}{What is a Neural Network ?}
\begin{figure}[!ht]
	\centering
	\rule{0cm}{0cm}
    \includegraphics[width=0.5\linewidth]{./images/ANN.png}
	\caption{A neural network}
\end{figure}
\end{frame}

\begin{frame}{What is a Neural Network?}
	A neuron is defined by :
	\begin{itemize}
		\item Bias
		\item Weight
		\item Activation function
	\end{itemize}
	\vspace{5mm}
	The bias and weight are updated during training. The activation function is chosen when the network is first designed.
	\begin{exampleblock}{Node output}
	\[f(\sum\limits_{i}W_{i}x_{i} + b)\]
	\end{exampleblock}
\end{frame}

\begin{frame}{Activation function}
	The activation function often if the sigmoid or hyperbolic tangent function.
	\begin{center}
		\includegraphics[scale=0.28]{images/sigmoid.png}
		\includegraphics[scale=0.28]{images/tanh.png}
	\end{center}
\end{frame}

\begin{frame}{Why Neural Network Language Modeling?}
	The state of the art uses n-grams
	\begin{itemize}
		\item Data spareness requires smoothing to avoid small or zero probabilities for some valid word sequences
		\item Discrete nature $\rightarrow$ generalization is hard. No notion of word similarity
	\end{itemize}
	\vspace{5mm}
	The neural network language model embeds words in a continuous space. The aim is to use appropriate data for training so that similar words are close in the continuous space, thus obtaining better results for unseen n-grams.
\end{frame}

\begin{frame}{Feed-forward NNLM architecture}

\begin{figure}[!htb]\centering
    \includegraphics[width=0.8\linewidth]{./images/architecture.png}
    \caption{Feed-forward neural network language model architecture}\label{diagram:architecture}
\end{figure}

\end{frame}


\begin{frame}{Feed-forward NNLM architecture}

\begin{exampleblock}{Neural network's operations}
\begin{equation}
\left.\begin{aligned}
    &d_j = tanh(\sum\limits_{l=1}^{\text{\tiny{$(n-1) \times P$}}}{M_{jl} \times c_l + b_j}) \qquad \text{\small{$\forall j = 1, \ldots, H$}}&\\
    &o_i = \sum\limits_{j=1}^{H}{V_{ij} \times d_j + k_i} \qquad \text{\small{$\forall i = 1, \ldots, N$}}&\\
    &p_i = \frac{exp(o_i)}{\sum\limits_{r=1}^{N}{exp(o_r)}} = P(w_j = i | h_j)&\\
\end{aligned}\right.
\end{equation}
\end{exampleblock}

where $M$ and $V$ are the \textbf{weight matrices} between the projection and hidden layers and between the hidden and output layers \\and $b_j$ and $k_i$ are the hidden and output layer \textbf{biases} respectively.

\end{frame}


\begin{frame}{Feed-forward Deep NNLM architecture}

The DNNLM has the same architecture as a NNLM but with \textbf{several hidden layers of nonlinearities}.
\newline
\newline
In the acoustic modeling community, DNNs have proven to be competitive with the well-established Gaussian mixture model:
\newline
\begin{center}
\textbf{How well DNNs can perform in language modeling?}
\end{center}

\end{frame}


\begin{frame}{Experimental setup}

\textbf{Baseline ASR system}
\newline
\newline
900K sentences (23.5M words) from 1993 WSJ text with verbalized punctuation from the CSR-III: \\
\begin{itemize}
    \item 2,439 unique utterances (46,888 words) as \textit{evaluation set}.
    \item 977 utterances (18,279 words) as \textit{development set}.
    \item The rest as \textit{training set}.\\
\end{itemize}

Comparison with the state-of-the-art \textbf{model M language model}.
\newline
\newline
After rescoring, the \textbf{baseline WER} is $20.7\%$ on the held-out set and $22.3\%$ on the test set.

\end{frame}

\begin{frame}{Experimental setup}

\textbf{DNN language model set-up}
\newline
\newline

\end{frame}

\begin{frame}{Results}
	\begin{figure}[!htb]
		\centering
	    \includegraphics[width=0.8\linewidth]{./images/results1.png}
	    \caption{WER function of number of hidden layers}
	\end{figure}
\end{frame}

\begin{frame}{Results}
	\begin{figure}[!htb]
		\centering
			\begin{minipage}{0.45\textwidth}
				    \includegraphics[width=\linewidth]{./images/results2.png}
			\end{minipage}
			\hspace{5mm}
			\begin{minipage}{0.45\textwidth}
				    \includegraphics[width=\linewidth]{./images/results3.png}
			\end{minipage}
	\end{figure}
\end{frame}

\begin{frame}{Results}
	\begin{figure}[!htb]
		\centering
			\begin{minipage}{0.45\textwidth}
				    \includegraphics[width=\linewidth]{./images/results4.png}
			\end{minipage}
			\hspace{5mm}
			\begin{minipage}{0.45\textwidth}
				    \includegraphics[width=\linewidth]{./images/results5.png}
			\end{minipage}
		\caption{WER function of number of hidden layers}
	\end{figure}
\end{frame}

\begin{frame}{Conclusion}
	\begin{itemize}
		\item DNNLM show an improvement over shallow NNLM
		\item DNNLM performances seem close to state of the art
		\item DNNLM should be compared to RNNLM
		\item Further research is needed
		\begin{itemize}
			\item Behavior with higher projection layer dimension is unclear
			\item Better pre-training and training strategies could be found
		\end{itemize}
	\end{itemize}
\end{frame}

\end{document}
